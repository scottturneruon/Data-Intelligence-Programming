\chapter{Playing with Pandas}
\section{You overall Task} 

To play and explore Panda and the DataFrame To produce a tutorial document to explain the experiments asked for and your own experimentation.

In groups of 2 or 3 people I want you to work together to produce your own version of the tutorial.

Rule: Do not move on to the next task until all members of the small group have finished.

\section{Task 1}

Create a new Jupyter Notebook for your tutorial (play with the markdown you met last week to improve the readability of the tutorial)

\begin{verbatim}
import pandas as pd
from matplotlib import pyplot as plt
df=pd.read_csv('covid_2.csv')
print(df.to_string())
data11=df.get('group')
data21=df.get('group2')
plt.plot(data11)
plt.plot(data21)
plt.show()
\end{verbatim}


Now imagine

Group2: Is incidences of COVID-19 during a short time period for people35 years old or older and are healthworkers

Group: Is incidences of COVID-19 during a short time period for people under 35 years old and are healthworkers 

Using markdown explain

(i) The data set to a reader, what is it essentially.

(ii) Any observations of the plots, do they show anything?

You can change the plots inline with what you did last week to make it easier to explain.

\section{Task 2}

What is a data-frame explain that somewhere or through the notebook using markdown.

\section{Task 3} 

Add these an explain what the do - please explain what each one does individually to add the reader.

(i) df.describe() #this might be of use 
\url{https://pandas.pydata.org/pandas-docs/stable/reference/api/pandas.DataFrame.describe.html}


(ii) df.pct_change() #this might be of use \url{https://www.tutorialspoint.com/python_pandas/python_pandas_statistical_functions.htm}

(iii) df.corr() #this might be of use
\url{https://www.tutorialspoint.com/python_pandas/python_pandas_statistical_functions.htm}

For each of these after the output has been produced using the markdown write down an explanation of what it shows.

\section{Task 4}

Using resources available for example but not exclusively

 Python Pandas - Statistical Functions

Python Pandas - Statistical Functions, Statistical methods help in the understanding and analyzing the behavior of data. We will now learn a few statistical functions, which we can apply on Pandas ob

  Python | Pandas DataFrame - GeeksforGeeks

A Computer Science portal for geeks. It contains well written, well thought and well explained computer science and programming articles, quizzes and practice/competitive programming/company interview Questions.

GeeksforGeeks

Find six methods relating to DataFrames not covered in the activities and for each

(i) Discuss in the group what the method does, what it shows about the data.

(ii) Using markdown and a description of what the method does in your own words but explaining it to another reader.

(iii) The code

(iv) Using markdown. What does the method reveal about the data?

\section{Task 5}

Individually go through your tutorial you have just produced. Now imagine you were writing this for a Government Minister what would you change? 

\section{Reflection}

If you were given a dataset to analyse, what do you think are the important things that should be in the notebook to make it accessible to the reader?

What do we have to think about when considering the accessibility of the notebook to the reader?

Why would be want to use DataFrames in Python? Why not stick to plain CSV? Does your notebook explain this?

\section{Can you did it your self}
Goal

Apply what you have learnt to 'real data'?

Task

(i) Go online and find some CSV file. It is up to what it is.possible starting point https://corgis-edu.github.io/corgis/csv/ (old but lots of data) WARNING these datasets are quite big you might want to find something smaller. Kaggle is good as well https://www.kaggle.com/datasets?fileType=csv

(ii) Load into Jupyter and create a new notebook.

(iii) Create a markdown entry to describe the dataset.

(iv) Write code to load in the new dataset into python.

(v) Use a combination of code and markdown; analysis and explain what it shows applying what you have done today

(vi) 10 minutes before the end of the activity (approx. 12:20) you must be ready to share your analysis with another member of the class.

Reflections

Did the other person understand and ask questions?

Can you use the questions asked to refine your analysis?

Can you think of better ways to visualise the results?

