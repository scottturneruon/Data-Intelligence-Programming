\chapter{Weather}
oGoals:

1. Provide an activity that can be used in your assignment

2. To play with bringing data from web sources.

Task 1

Go to https://openweathermap.org/api/ and current weather data.

Subscribe for free access and get Api. You will need to provide an email address sometime it is better to have a sacrificial email account to use for these kind of activities.

Copy the API key you get, you will need to use it later.

Task 2

Download this notebook 

WeatherStation_class.ipynb
 . We are going to use a modified version of "Project: Fetching Current Weather Data" from "Automate the boring stuff with Python" by Al Sweigart e-book available at https://ebookcentral.proquest.com/lib/canterburychristchurch/detail.action?docID=4503140 pages 383-387.

If you want to break the code up into explainable chunks in the notebook and describe what each does using markdown.

If you were do this for your assignment the minimum expectation to be done (and I know you all want to better than that :-) ) for "all the questions and tasks" described in the marking scheme is the following.

All of task 1
All expanded and improved the explanation in the notebook including how is the data shown been accessed to aid the reader.
The last part of the code prints out temperature in Kelvin change it to print out the temperature instead in Centigrade and Fahrenheit - at the moment the Kelvin value is a string [look back into earlier examples of how we dealt with that problem]
In the notebook a reference list in markdown and some citation/references in the main text. As an example where did you get the temperature conversions from that could be referenced.
Ideas to take it further if used in the assignment

The examples below are not requirements but some ideas to help. In all cases clear and read documentation with lots of analysis is going to help.

 

1. Get the system to collect and print out weather information from 5 cities automatically

 

2. Get the system to collect information automatically from a number of cities and do compare and contrast to make recommendations to the user. Use https://openweathermap.org/ to search for other cities to include.

 

3. Combine this API with data from others APIs to extend what the software does. You will have to get a new API key for the new APIs you use.

 

4. You could also think about how to combine the outputs graphically in table form, etc.

sources of useful information

//openweathermap.org/api/

https://ebookcentral.proquest.com/lib/canterburychristchurch/detail.action?docID=4503140

