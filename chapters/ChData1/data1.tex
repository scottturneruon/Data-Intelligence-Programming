\chapter{Staert of playing with data}
Purpose to investigate simple plots using the matplotlib module and combine that with using CSV file handling

\section{Lets Play}
(i) Create a new Juypter Notebook for this task.

(ii) Using the code and the markdown make a tutorial to explain the following code:

- break the code up into chunks to explain what is happening what the code does

-you can use this file for the CSV file

covid_1.csv


- Think about how you would explain what is going on to another person.

- Now using either the discussion board or speaking in your group explain what is happening.

import csvfrom matplotlib import pyplot as pltdata1=[]data2= []data0= []with open('covid_1.csv', 'r') as file:reader = csv.reader(file)for row in reader:data1.append(float(row[1]))data2.append(float(row[2]))data0.append(int(row[0]))file.close()plt.plot(data0,data1)plt.plot(data0,data2)plt.show()
Task 2

(i) Replace the lines marked with # below with the line not marked with #

#plt.plot(data0,data1)#plt.plot(data0,data2)plt.plot(data0, data1,'ro',data0,data2,'b.-')plt.show()
(ii) Describe what the new line does.

(iii) Experiment with different parameters in the ' ' sections of the line, change colours and shapes.

(iv) Using https://www.w3schools.com/python/matplotlib_labels.asp as a reference source and axes labels and titles to your plot.

(v) Using https://www.w3schools.com/python/matplotlib_line.asp as a reference source experiments with different styles of line.

Task 3

Using https://www.w3schools.com/python/matplotlib_intro.asp as your starting point and all the links to MATPLOTLIB in the left-hand column add more features to the graphs for examples grids, markers and anything you feel is appropriate to make the graphs more informative. You could try subplots for example.

Task 4

(i) In your notebook add this code

import numpy as npx = np.random.normal(170, 10, 250)plt.hist(x)plt.show()
(ii) numpy is another module we use in data science https://www.w3schools.com/python/numpy/default.asp for details. Allow us to add in arrays, probability distributions, etc. Taking what you have learn in task 3 and appropriate axes labels etc to make the graph much more understandable.

Thought 

It is usually your job as a Data Scientist to present the information in as accessible a form a possible, often your main job along with analysing the data is to get the data presented in a format that is most accessible to the reader.

\section{Lets play 2}
Using this example from https://oxylabs.io/blog/python-parse-json#Converting-JSON-string-to-Python-object

import json# Tuple is encoded to JSON array.languages = ("English", "French")# Dictionary is encoded to JSON object.country = {"name": "Canada","population": 37742154,"languages": languages,"president": None,}with open('countries_exported.json', 'w') as f:json.dump(country, f)
Create a jupyter notebook that is a tutorial on what this does? As a minimum explain

What the different collections used are?

What dump does and what the effect of the final two lines?

What does with open do and why do we use it?

Task 2

In your notebook add

print(print(json.dumps(country, indent=4)) )
Explain in the notebook what the difference between dump() and dumps() is.

Experiment with ident=4 what does it do, include this in your notebook

Task 3

Create code replicating some of the features in task 1 and 2 but for a JSON formatted data called student containing:

-name

-id number

-the three modules they are studying this semester

-the three modules they are studying next semester.

Save as JSON file student.json

Task 4

The Challenge: It is involves reading around ! Add to your notebook(s) code to read in a JSON file and display its contents.

