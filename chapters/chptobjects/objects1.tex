\chapter{Objects}
Everything in Python is an Object. That is a bold statement and what does it actually mean?

\section{inheriting}

\begin{lstlisting}
class player(person):
  pass
\end{lstlisting}
All the attributes and methods from person class but don’t want to add more –use pass
\begin{lstlisting}
class player(person): 
 def __init__(self, name): 
   super().__init__(name)

\end{lstlisting}
\section{Putting it together}
But now we can also add methods and attributes and over-write methods
\newline
Let's put add in the def to make a more useful class
\begin{lstlisting}
    class room:
      def __init__(self, name, description):
        self.name = name
        self.description = description
      def displayRoom(self):
        print(self.name)
      def set_description(self, room_description):
        self.description = room_description
      def get_name(self, name):
        return self.name
\end{lstlisting}
Even though this is a fairly simple bit of code it has a lot of the basic elements we need.
\subsection{def __init__()}
\begin{lstlisting}
    def __init__(self, name, description):
        self.name = name
        self.description = description
\end{lstlisting}
All of our class need something to initialise them this the def __init__
When we set up an instance of the class it sets up the starting point for the instance, so 
In the case above the name and description associated with the instances and stores them as name and description. So if set up an instance of room called hall then in our program the description associated with hall would be stored in description.
\subsection{adding methods}
As well as storing data we want methods (see earlier chapter) to carry out operations
\begin{lstlisting}
      def displayRoom(self):
        print(self.name)
      def set_description(self, room_description):
        self.description = room_description
      def get_name(self, name):
        return self.name
\end{lstlisting}
All, as in the previous chapter, are defined using def().
\newline
We have three methods defined here
\begin{verbatim}
    displayRoom() prints out the name of the room on the screen
    set_description() changes or sets the description based on the description passed into it
    get_name() pass back to the program what is stored in name or it gets the name  
\end{verbatim}

You must get into the habit of naming methods that change the data/attribute as starting with the set and those that obtain the data/attribute contents as starting with get. This is a widely used convention and one that you are expacted to use.
