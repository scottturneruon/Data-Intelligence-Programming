
\chapter{Basic Concepts}
\section{Introduction }
This is an opportunity to learn how to computer program for data intelligent and data science systems. How to critically apply mathematics and everyday heuristics into the fundamentals of computer programming concepts to achieve functional and useable computer application. The module learning is building block of learning to support future learning in design and development of data intelligent systems.

\begin{itemize}
\item Python Programming (and Programming in general)
\item Applying it to problems.
\item Documenting it appropriately for Data Scientists
\item Picking up a few tools on the way
\end{itemize}


\section{Mistakes, Simplicity and Errors}
You will make mistakes! We all do. The trick is to not let them stop you and learn from them.
\begin{itemize}
    \item F First
    \item A Attempt at
    \item I Initial
    \item L Learning
\end{itemize}\par
It might seem strange to talk about failure and mistakes this early on but actually they can stop you learning. You will make mistakes we all do, but think of it as a learning point not a disaster. Programming is one of those topics that is best learnt with an element of Trial and Error - "I know how to do this, but what happens if I change this?" or "I want to do this, so how can I do it?"
\par
Everything should be a simple as possible, making things complicated is easy.
\begin{itemize}
    \item “If you can't explain it to a six year old, you don't understand it yourself.” Albert Einstein
    \item “It is always the simple that produces the marvelous.” Amelia Barr

\end{itemize}

\section{Three of Four Constructs}
At its simplest programming is made up of four basic structures or constructs.
\begin{itemize}
    \item Sequence
    \item Conditional statements, decisions
    \item Iteration, looping and repeating
    \item subroutines, methods, classes, objects...
\end{itemize}

Before we start a few things to remember, that I find useful when doing these types of activities:
\begin{itemize}
    \item Computers are not intelligent; we have to tell it literally what to do.
    \item You're the boss, you tell the machine what to do not the other way around.
    \item You are not writing the instructions for you do to the activity but for someone else. They must be able to do follow these without coming back to you and asking "What did you mean here?"
    \item There is not one answer to these activity, your solution and another group's can both be right.  
    \item We need test and find things we have missed - it is not personal.
\end{itemize}

So let's look at our three tools using an example of opening a bottle.

A sequence of instructions
\begin{lstlisting}
Find the lid of a bottle
Turn lid of the bottle until lid comes off
\end{lstlisting}

A loop to repeat a sequence until something we define happens or even to continue forever. So our instruction "Turn lid of the bottle until lid comes off" might be changed to break it down further to produce a certain sequence of instructions to be repeated. I am going use indenting to show what is inside the loop (the bit that repeats).
\begin{lstlisting}
hold the bottle
Repeat until lid comes off
   turn the lid 45 degrees anti-clockwise
   release lid hand 
   move lid hand back starting position
   check does the lid come off?
\end{lstlisting}

Our last tool is a test (conditional statement) where we can change what happens based on the outcome of the test. Again I i am going to use indenting to show what belongs togther. In the previous 'code' we had a line saying "check does the lid come off?" which was two outcome if the lid does come off or it doesn't, This sounds like a test

\begin{lstlisting}
if check does the lid come off == yes
   pull the lid vertical 10 cm
   stop the algorithm
\end{lstlisting}

So lets put all of these together
\begin{lstlisting}
Find the lid of a bottle
hold the bottle
Repeat until lid comes off
   turn the lid 45 degrees anti-clockwise
   release lid hand 
   move lid hand back starting position
   if check does the lid come off == yes
     pull the lid vertical 10 cm
     stop the algorithm
\end{lstlisting}

So now we have a starting point for a routine/algorithm for opening a bottle.

\section{Activity - Make your own Art game}
This activity is about using computing ideas to build a paper-based game. Aim is to create pictures by developing and then following a set of instructions (Algorithms). We are going to use the three tools to make the games. 

Goal 1: At the end of the session you have produce a paper-based game's instructions that use dice, pens and paper to produce random drawings (see Figure 5.1). The instructions will probably look a bit like the bottle opening routine 

Goal 2: You have tested some one else's game and constructively feedback to the developers, ideas, improvements, etc. Testing is an important part of Computing.


Task: \textbf{Develop in groups of 2-3 people, a paper-based game's instructions that use dice, pens and paper to produce random drawings. It must use the 'Three Tools', dice, pens and paper; BUT how you use them is up to you. Your instructions will be in sentences so the user can understand exactly what you mean. Your group must test another group's game and provide constructive help.} 


You have two dice, some pens and paper.

First 20 minutes
Working in groups of 2-3 people, develop your games and produce the instructions.

Remaining Time
Swap your game with another group and each try their game. Pass back constructive feedback to the other group. Some suggestions to possibly ask yourself when testing.
\begin{itemize}
    \item Could a particular instruction be viewed as meaning something else?
    \item Is there something the developer's could add to improve the game in your opinion?
\end{itemize}

Don't panic: In case of emergency in the next section there is an example, please build your own first.
\newline
\newline
section{First Python program}
We are going to start with a sequence.

\subsection{Helpful bits}
We need to understand a few things.\par
The program will start with the line import turtle all this does, for the moment, is add some instructions do make an on screen graphic move and draw - the turtle.\par
Next bit we need is the line scrn=turtle.getscreen() and all that does is connect the turtle to the screen.\par
Because we could have more than one turtle we have give it a 'name' to use it that is done with firstturtle =turtle.Turtle(). \par
Now we need to make the turtle do something. So let's make the turtle move 100 pixels on the screen in the directions it is facing. We can do that with this line firstturtle.forward(100) which we can think of as saying turtle called firstturtle move forward 100 pixels.\par
One more action firstturtle.right(90) - right(90) means turn the turtle called firstturtle 90 degrees to the right. Any guesses what firstturtle.left(90) would do?


\subsection{The code}
\begin{lstlisting}
import turtle
scrn=turtle.getscreen()
firstturtle =turtle.Turtle()
firstturtle.forward(100)
firstturtle.right(90)
firstturtle.forward(100)
firstturtle.right(90)
firstturtle.forward(100)
firstturtle.right(90)
firstturtle.forward(100)
firstturtle.right(90)
\end{lstlisting}

\subsection{make and modify}
Ok now adapt it.\par 
Change the program so that it draws a triangle


\newpage
\section{Some solutions ideas}
\subsection{Thomas' Tangles}
This is an example of the type of solution that could be built. Aim for your's to be better than this one.

Using crayons, pencils or pens, we are going to follow an algorithm to create a random drawing. This could be done in pairs and you will need squared paper. 

Person A: Rolls the dice and reads out the instructions - their role is to roll the dice, interpret the algorithm and tell the 'robot' what to do.

Person B: Is the ‘robot carrying out the instructions'. The lines are solid blocks of colour so move four squares does also mean colour in the squares between the start and finish in the direction of movement.

\begin{figure}
    \centering
    \includegraphics[width=10cm]{chapters/chapter1/figures/tt1.JPG}
    \caption{Thomas' Tangles}
    \label{fig:ThomasTangles1}
\end{figure}

When a new central square is needed the roles of A and B swap (so A is the ‘robot’ and B rolls the dice and reads out the instruction). The roles keep swapping.

\begin{lstlisting}
Start from a random square – call it the centre square
Repeat until end of game
If die roll = 1
  Roll die for number of moves
 move die roll number of steps up the page
If die roll = 2
  Roll die for number of moves
  move die roll number of steps down the page
If die roll = 3
  Roll die for number of moves
  move die roll number of steps to the left 
If die roll = 4
  Roll die for number of moves
  move die roll number of steps to the right
If die roll = 5
  Roll die
  If die = 1 change colour to Red
  If die = 2 change colour to Blue
      If die = 3 change colour to Black
  If die = 4 change colour to Green
  If die = 5 change colour to Orange
  If die = 6 change colour to Yellow
If die roll = 6
           Roll die
           Return to current centre square
           If the second die roll=6
                   randomly select new centre square
                   if block is off the page
                      randomly select new centre square
\end{lstlisting}

The Scratch version can be here \url{https://scratch.mit.edu/projects/135816631/} if you wish to see the code.






