\chapter{Methods}
So in the last two chapters we have looked at looping, making decisions and sequences and we started work on our games.\paragraph{}
So we left the game
\begin{lstlisting}
room="hall“
while True:
  room=input("Which room? ")
  if room=="hall" or room=="Hall":
     print("you are in the hall - there are two rooms off this\n -A kitchen\n-A study\n")
  elif room=="kitchen":
      print("your are in the kitchen")
  elif room=="study": 
     print("you are in the Study")
  else:
      print("You have entered the Twilight Zone and are lost in space and Time")

\end{lstlisting}
It is getting better but we still have some issues.\paragraph{}
We can jump from kitchen to study without going through the hall.
What if we want to have rooms coming of the kitchen to other rooms?
Just a few issues.
\section{define it}
So what if we made the rooms it a ‘thing’ and prints messages but also helps us structure what happens next. So what if we start defining what each room is. I am going to sneek in inputs as well. So we have a way of defining the room.
\begin{lstlisting}
def hall():
  print("you are in the hall - there are two rooms off this\n -A kitchen\n-A study\n")
  room=input("Which room? ")
  if room=="kitchen":
    kitchen()
  elif room=="study":
    study()
  else:
    print("You have entered the Twilight Zone and are lost in space and Time")

def kitchen():
  print("your are in the kitchen, there is nothing in here") 
  print("You leave this room and back to the Hall")
  hall()  

def study():
  print("you are in the Study")
  print("Back to the hall")
  hall()
  
while True:
    hall()
\end{lstlisting}
You always start in the hall
When you leave the kitchen and study you must go back to the hall - this is has just been done to make the changes easier for the moment.\paragraph{}
So what have we done with these def lines is define what happens when we are in a room. We can change what happens in a room without necessarily altering what happens in other rooms. So if I looking for what happens in the kitchen I only need to go and look at:
\begin{lstlisting}
def kitchen():
  print("your are in the kitchen, there is nothing in here") 
  print("You leave this room and back to the Hall")
  hall() 
\end{lstlisting}
These defs are subroutines.

\section{local and global}
So lets add in an item so we can pick it up or put it down.

\begin{lstlisting}
item ="No items"

def hall():
  print("you are in the hall - there are two rooms off this\n -A kitchen\n-A study\n")
  room=input("Which room? ")
  if room=="kitchen":
    kitchen()
  elif room=="study":
    study()
  else:
    print("You have entered the Twilight Zone and are lost in space and Time")

def kitchen():
  #global variable which means change it here change it everywhere
  global item
  item="a knife"
  #when printing out strings we can use the + to
  #combine them and # is for comments
  print("your are in the kitchen, there is "+item) 
  print("You leave this room and back to the Hall")
  hall()  

def study():
  print("You are carrying "+item)
  print("you are in the Study")
  print("Back to the hall")
  hall()
  
while True:
    hall()
\end{lstlisting}
If we had created item inside one of the defs it could only we seen within the define not outside of it. So we need in this case to have a variable item that is seen and can be changed anywhere that is the global bit seen in kitchen():