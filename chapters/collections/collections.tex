\chapter{Collections}
\section{The List}
Arrays as such are not in standard Python but are available in other libraries. First storage method is the list.\paragraph{}
These are  Ordered; Allow duplicate values; Changeable;Indexed (first item [0], second [1], etc); item Can have different data types


\begin{lstlisting}
products = ["flour","apple","punnet of raspberries","bread", "apple"]
print(products) #prints out all the products ['flour', 'apple', 'punnet of raspberries', 'bread', 'apple'
print(len(products)) #prints out the number of elements in the list products
product_detail = ["flour", 3.59,"aisle 1","storeroom aisle A", 34]
print(product_detail)
print("item: "+product_detail[0]+" costs: £"
+str(product_detail[1])+" can be found at "+product_detail[2])
print("item: "+product_detail[0]+" costs: £",product_detail[1]," can be found at "+product_detail[2])
print("item: "+product_detail[0]+" costs: £"+str(product_detail[1])+" can be found at "+product_detail[3]+" there are "+str(product_detail[4])+" in stock")

\end{lstlisting}

\section{Loooping in collections}
We can loop through lists, etc in the same way – replace thislist with tuple etc. In other words any of the collections.• 
\begin{lstlisting}
    thislist = ["apple", "banana", "cherry"]
    for x in thislist:  
        print(x
\end{lstlisting}

\section{Tuples}
Multiple items in a single variable
• Ordered
• Unchangable – can’t change, add or remove items once created – directly (can do it by making it a list and converting back though) - immutable
• List uses [] Tuple use ()
• Indexed

\begin{lstlisting}
    thislist = ("apple", "banana", "cherry")
    for x in thislist:  
        print(x
\end{lstlisting}

\section{Sets}
Unordered
Unchangeable sort of
Unindexed
uses {}
No duplicates

You can’t change a set but you can add using add() method for example products.add("dog")

Two ways to remove an item from the set Pop() removes the last item in the set warning: not necessarily the last item put into the set e.g. product.pop()– it is unordered Remove() removes the item specified for example product.remove("bread")

\section{Dictionaries}
A little different• Store data values as key: value pairs• So lets return
\newline
\begin{lstlisting}
    product_detail = {"flour", 3.59,"aisle 1","storeroom aisle A", 34}
\end{lstlisting}

\newline 
How could a dictionary help? Well it could give meaning a bit• We have actual used this in our murder-mystery games in the room class

As an example
\begin{lstlisting}
    product_detail = {
     "product":"flour",
    "price":3.59,
    "store location":"aisle 1",
    "storeroom location":"storeroom aisle A",
    "no. in stock":34• }
\end{lstlisting}
The code
\begin{lstlisting}
    print(product_detail)
    print(product_detail["product"])
    print(product_detail["price"])
\end{lstlisting}
Produces
\begin{verbatim}
    {'product': 'flour', 'price': 3.59, 'store location': 'aisle 1','storeroom location': 'storeroom aisle A', 'no. in stock': 34}
    flour
    3.59
\end{verbatim}